\documentclass{article}

% Set page size and margins
\usepackage[a4paper,top=2cm,bottom=2cm,left=2.5cm,right=2.5cm,marginparwidth=1.75cm]{geometry}

% Useful packages

\usepackage{graphicx}
%\usepackage[draft]{graphicx}

\usepackage{amsmath}
\usepackage{wrapfig}
\usepackage{subcaption}
\usepackage{caption}
\usepackage{float}
\usepackage{multicol}
\usepackage[colorlinks=true, allcolors=blue]{hyperref}
\usepackage{lscape} 

\renewcommand\arraystretch{1.25}

\usepackage{listings}
\usepackage{color}
\usepackage[percent]{overpic}

\definecolor{dkgreen}{rgb}{0,0.6,0}
\definecolor{gray}{rgb}{0.5,0.5,0.5}
\definecolor{mauve}{rgb}{0.58,0,0.82}

\lstset{frame=tb,
  language=R,
  aboveskip=3mm,
  belowskip=3mm,
  showstringspaces=false,
  columns=flexible,
  basicstyle={\small\ttfamily},
  numbers=none,
  numberstyle=\tiny\color{gray},
  keywordstyle=\color{blue},
  commentstyle=\color{dkgreen},
  stringstyle=\color{mauve},
  breaklines=true,
  breakatwhitespace=true,
  tabsize=3
}

%\numberwithout{figure}{section}
\counterwithout*{figure}{section} % this is the only way i got the figure number and in-text reference to match, no idea what kind of black magic is happening here though


\title{How many genes are in a gene family? Uniform annotation methods for comparative genomics}
\author{Milena Trabert, Lila Maladesky, Elina Immonen}
\date{May 2025}

\begin{document}
\renewcommand{\arraystretch}{1.5}
\pagenumbering{gobble}
\renewcommand{\thetable}{S\arabic{table}}
\renewcommand{\thefigure}{S\arabic{figure}}


\section*{Supplementary information}

\subsection*{Genome annotation}

\begin{table}[H]
    \centering
    \caption{BUSCO scores for \textit{C. magnifica}, \textit{T. freemani} and \textit{C. maculatus} genome annotations}
    \label{SItab:BUSCO}
    \begin{tabular}{l |c |c | c}
        BUSCO category & \textit{C. magnifica} &  \textit{T. freemani} & \textit{C. maculatus} \\
        \hline
        \hline	   
        Complete BUSCOs & 972 (96.0\%) & 978 (96.6\%) & 999 (98.7\%) \\
        \hline
        Complete and single-copy BUSCOs & 877 (86.6\%) & 896 (88.5\%) & 731 (72.2\%) \\
        \hline
        Complete and duplicated BUSCOs & 95	(9.4\%) & 82 (8.1\%) & 268 (26.5\%) \\
        \hline
        Fragmented BUSCOs & 9 (0.9\%) & 14 (1.4\%) & 6 (0.6\%) \\
        \hline
        Missing BUSCOs & 32	(3.1\%) & 21 (2.0\%) & 8 (0.7\%) \\
        \hline
        Total BUSCO groups searched & 1013	 & 1013 & 1013 \\
    \end{tabular}
\end{table}

\begin{figure}[H]
    \centering
    \includegraphics[width=0.75\linewidth]{../data/annotation_evaluation/single_exon_genes_white_bg.png}
    \caption{Number of genes with highlighted proportion of single exon genes in all species included.}
    \label{SIfig:gene_nos}
\end{figure}

\subsection*{Synteny}

\subsubsection*{\textit{B. siliquastri} vs. \textit{A. obtectus}}

\begin{figure}[H]
    \centering
    \includegraphics[width=\linewidth]{../data/synteny_plots/plot_ctl_Aobt_Bsil.png}
    \caption{Synteny plot of \textit{B. siliquastri} (left) and \textit{A. obtectus} (right), the X-chromosomes are bs9 for \textit{B. siliquastri} 
    and ao1950 for \textit{A. obtectus} (second from the bottom).}
    \label{SIfig:synteny_BsilAobt}
\end{figure}

\subsubsection*{\textit{C. maculatus} vs. \textit{A. obtectus}}

\begin{figure}[H]
    \centering
    \includegraphics[width=\linewidth]{../data/synteny_plots/plot_ctl_Cmac_Aobt.png}
    \caption{Synteny plot of \textit{A. obtectus} (left) and \textit{C. maculatus} (right), the X-chromosomes are cm124 for \textit{C. maculatus} 
    and ao1950 for \textit{A. obtectus} (second from the top).}
    \label{SIfig:synteny_CmacAobt}
\end{figure}

\subsubsection*{\textit{C. magnifica} vs. \textit{C. septempunctata}}

\begin{figure}[H]
    \centering
    \includegraphics[width=\linewidth]{../data/synteny_plots/plot_ctl_Csep_Cmag.png}
    \caption{Synteny plot of \textit{C. magnifica} (left) and \textit{C. septempunctata} (right), the X-chromosomes are mc64 for \textit{C. magnifica} 
    and cs7 for \textit{C. septempunctata} (second from the bottom).}
    \label{SIfig:synteny_CmagCsep}
\end{figure}

\subsubsection*{\textit{T. castaneum} vs. \textit{T. freemani}}

\begin{figure}[H]
    \centering
    \includegraphics[width=\linewidth]{../data/synteny_plots/plot_ctl_Tcas_Tfre.png}
    \caption{Synteny plot of \textit{T. castaneum} (left) and \textit{T. freemani} (right), the X-chromosomes are tc36 for \textit{T. castaneum} 
    and tf4 for \textit{T. freemani} (first from the bottom).}
    \label{SIfig:synteny_TcasTfre}
\end{figure}

\subsubsection*{\textit{B. siliquastri} vs. \textit{T. castaneum}}

\begin{figure}[H]
    \centering
    \includegraphics[width=\linewidth]{../data/synteny_plots/plot_ctl_Tcas_Bsil.png}
    \caption{Synteny plot of \textit{B. siliquastri} (right) and \textit{T. castaneum} (right), the X-chromosomes are bs9 for \textit{B. siliquastri} 
    and tc36 for \textit{T. castaneum} (third and second from the bottom).}
    \label{SIfig:synteny_TcasBsil}
\end{figure}


\end{document}